\documentclass[12pt]{article}
\usepackage{graphicx, amssymb}
\usepackage{mdframed}
\usepackage[margin=1in]{geometry}
\begin{document}

\begin{mdframed}
  \begin{center}{\huge Stat 139: Project Proposal}\\
    {\large Death and Mortality in the US}\\
Afshine Amidi, Michael Ge, Peter Jin, Xingchi Dai\end{center}
\end{mdframed}

With various medical advancements and increasing standards of living in the last century, the human lifespan has increased to unprecedented heights. Issues of meeting basic living conditions are now diminished in exchange for more difficult diseases that appear only later in life. This paper seeks to examine the ways in which death and mortality affects humans in the modern day. By the end of our research, we hope to have a better understanding of the most challenging and urgent problems the world faces.\\

To limit the scope of the paper, we will examine statistics only for the United States. The project will be broken into several parts. First, we will have a cursory overview of the general mortality rate in the US, which we predict will be decreasing over the years. In this analysis, we will begin to understand the major sources of mortality that now affect the US more or less than it had before.\\

With this picture in mind, we will then delve into some of the more impactful metrics to understand their change. For example, we will expect diseases like cancer or other chronic/genetic diseases will increase, while infectious diseases will decrease. As a result, we will take a look at various metrics that might affect each category. For example, we will look at the change in spending for infectious disease research over time, the use quantity of vaccines over the years, and other metrics for infectious diseases, while we will look at similar metrics for non-infectious diseases. Hopefully, we will be able to find a relationship between a lack of funding or a lack of public action against a certain disease that will highlight a need to focus in that area.\\

Finally, after understanding the diseases or other sources of death that cause the most mortality, we will then know which sources of mortality are the highest priority to be solved. Using these findings, and given that the project is not already too large in scope, we will then examine data sets that might indicate ways in which the US might be able to improve its focus on these diseases.\\

So far, we have done preliminary research, finding datasets from healthdata, the CDC, and other public data sites, to collect possibly useful information for the project. From there, we will have to figure out the most efficient way to tackle this large problem by figuring out which datasets will be used for which portion of the research, and what exact findings we will hope to report in the final paper. Challenges we foresee we will face is understanding all the data we find, and how to use each one together using the various techniques we learned in class.
\end{document}